\section{Summary}
The main result from this chapter was the explanation of three engine cycles: Stirling, Otto, and Diesel.  Each of these cycles can convert excess heat, typically from combustion, to work.  The Stirling cycle can also use work in order to transfer heat from a cold environment to a hot environment, thus acting as either a heater or a refrigerator.  All of these cycles typically use ideal gases as the working fluid, so the ideal gas law was used in abundance.

In order to analyze these cycles, we introduced the First Law of Thermodynamics:
\begin{equation*}
  \Delta U = Q - W,
\end{equation*}
which is a statement of the conservation of energy.  The change in internal energy can be determined from the temperature change between states:
\begin{align*}
  \Delta U = U_2 - U_1 &= m c_v (T_2 - T_1) & \Delta u = u_2 - u_1 &= c_v (T_2 - T_1)
\end{align*}
The work is typically found through integration, which results in the following for special cases:
\begin{align*}
  {\rm Isobaric} & &\rightarrow& & W &= mp (v_2 - v_1) = p(V_2 - V_1)\\
  {\rm Isothermal} & &\rightarrow& & W &= mRT \ln \left(\frac{v_2}{v_1}\right)  = mRT \ln \left(\frac{V_2}{V_1}\right)\\
  {\rm Adiabatic} & &\rightarrow& & W &= m\:\frac{p_1 v_1 - p_2 v_2}{\gamma - 1} = m c_v (T_2-T_1)\\
  {\rm Isometric} & &\rightarrow& & W &= 0
\end{align*}

For the special case of constant pressure, the heat transfer for a process can be found using the constant pressure specific heat, $c_p$:

\begin{equation*}
  Q = m c_p (T_2 - T_1)
\end{equation*}

In all other circumstances, the heat transfer can only be determined through the First Law.

Finally, because two of the cycles use adiabatic compression/expansion, it was necessary to develop equations for the properties.  These are summarized below:

\begin{align*}
  \frac{p_2}{p_1} &= \left(\frac{T_2}{T_1}\right)^{\frac{\gamma}{\gamma -1}} = \left(\frac{v_1}{v_2}\right)^{\gamma} \\
  \frac{T_2}{T_1} &= \left(\frac{p_2}{p_1}\right)^{\frac{\gamma-1}{\gamma}} = \left(\frac{v_1}{v_2}\right)^{(\gamma -1)} \\
  \frac{v_2}{v_1} &= \left(\frac{p_1}{p_2}\right)^{\frac{1}{\gamma}} = \left(\frac{T_1}{T_2}\right)^{\frac{1}{(\gamma -1)}}
\end{align*}
