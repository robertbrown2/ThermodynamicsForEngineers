\begin{homework}
  %{\bf Conceptual Questions}
  % Section 3.1 - First Law
  \question Look at Figure \ref{fig:ch3_pV}.  Which process would require the most work to complete?  What does this say about the heat transfer during that process?
  
  \question Why is work negative in the First Law of Thermodynamics?

  % Section 3.2 - Enthalpy
  \question In what circumstances is enthalpy useful?  In other words, why would we use enthalpy instead of internal energy?
  
  \question Which is larger, internal energy or enthalpy?  Is this always true?  Why?

  % Section 3.3 - Specific Heats
  \question In what ways is specific heat different between gases and solids?
  
  \question In words, what is $c_p$?

  % Section 3.4 - Stirling Cycle Engine
  \question What is the function of the regenerator?  Without a regenerator, what would be the effect on the engine?
  
  \question Is there heating that occurs during constant temperature expansion/compression?  If so, how much?
  
  % Section 3.5 - Stirling Cycle Cooling
  \question Draw the cycles in a $p$-$v$ diagram for the Stirling Cycle Engine and Stirling Cycle Cooler.  What is different between the two? Label all heat flow and work in and out of the the system.  Use $Q_H$, $Q_C$, $Q_R$, $W_{in}$, and $W_{out}$ for your labels.

  % Section 3.6 - Ideal Gas Adiabatic Processes
  \question What is $\gamma$?  How does it change with temperature (see Table \ref{sec:idealGasAir})?
  
  \question What assumptions are necessary to use Equations \ref{eq:ch3_adiabaticIdealGasTV}, \ref {eq:ch3_adiabaticIdealGaspV}, and \ref{eq:ch3_adiabaticIdealGaspT}?
  
  % Section 3.7 - Otto Cycle
  \question How does the Otto cycle differ from the Stirling cycle?
  
  \question What parameters determine the efficiency of the Otto cycle?
  
  % Section 3.8 - Diesel Cycle
  \question How does the Diesel cycle differ from the Stirling and Otto cycles?
  
  \question Why does the Diesel cycle typically use a higher compression ratio than the Otto cycle?


  % Work-out Questions
  %{\bf Work-out Questions}
  % Section 3.1 - 1st Law
  \question Four kilograms of water at 50°C are placed in a piston cylinder device under 6 MPa of pressure.  Heat is added to the water at constant pressure until the temperature of the steam reaches 400°C.
  \begin{questionparts}
  \item Determine the work done by the fluid ($W$) \answer{[1.1 kJ]} and the heat transferred to the fluid ($Q$) \answer{[11.9 kJ]} during this process. 
  \item Plot the process on a $p$-$v$ diagram.
  \end{questionparts}
  \newpage
  % Section 3.2 - Enthalpy
  \question For each of the following states, use the steam tables (or CoolProp) to check the relationship $h=u+pv$:
  \begin{questionparts}
  \item Steam at 300°C and a pressure of 200 kPa \\ \answer{[3072.1 kJ/kg = 2808.8 kJ/kg + 200 kPa $\cdot$ 1.3162 $\rm m^3/kg$]}
  \item R-134a at a specific volume of 1 $\rm m^3$/kg and an internal energy of 200 kJ/kg \answer{[206.5 kJ/kg]}
  \item Carbon dioxide at a temperature of -50°C and a pressure of 100 kPa (use CoolProp) \answer{[444.7 kJ/kg]}
  \item Helium at a temperature of -150°C and a pressure of 5 MPa (use CoolProp)\answer{[658.4 kJ/kg]}
  \end{questionparts}
  
  % Section 3.3 - Specific Heats
  \question How much heat is required to raise the temperature of Helium by 50°C at constant pressure? \answer{[259.7 kJ/kg]} What about at constant volume? \answer{[155.8 kJ/kg]}
  
  % Section 3.4 - Stirling Engine
  \question Analyze a Stirling cycle engine which uses air as a working fluid.  Let the minimum pressure be 400 kPa, the maximum volume be 1000 $\rm cm^3$, and the minimum volume be 400 $\rm cm^3$.  The temperature of the hot side is 100°C, and the temperature of the cold side is 20°C.
  \begin{questionparts}
  \item Sketch each process and state on a $p$-$v$ diagram.
  \item Find the pressure at each state. \answer{[$p_1 = 400 \rm kPa$, $p_2 = 1000 \rm kPa$, $p_3 = 1273 \rm kPa$, $p_4 = 509 \rm kPa$]}
  \item Create a table with the pressure, temperature, and specific volume at each state.
  \item Find the change of internal energy, work, and heat transfer for each process.  Put this information in a table.
  \item Find the net work of a cycle \answer{[100 J]}, the input heat transfer \answer{[466 J]}, and the thermal efficiency \answer{[21\%]}.
  \end{questionparts}
  \newpage
  %
  % Section 3.5 - Stirling Cooling
   \question Analyze a Stirling cycle cooler which uses helium as a working fluid.  Let the minimum pressure be 800 kPa, the maximum volume be 200 $\rm cm^3$, and the minimum volume be 150 $\rm cm^3$.  The temperature of the hot side is 20°C, and the temperature of the cold side is -30°C.
  \begin{questionparts}
  \item Sketch each process and state on a $p$-$v$ diagram.
  \item Find the pressure at each state. \answer{[$p_1 = 965 \rm kPa$, $p_2 = 1286 \rm kPa$, $p_3 = 1067 \rm kPa$, $p_4 = 800 \rm kPa$]}
  \item Create a table with the pressure, temperature, and specific volume at each state.
  \item Find the change of internal energy, work, and heat transfer for each process.  Put this information in a table.
  \item Find the net work per cycle \answer{[-9.5 J]}, the heat transfer from the refrigerated side \answer{[46 J]}, and the coefficient of performance \answer{[4.86]}.
  \end{questionparts}
  %
  % Section 3.6 - Ideal Gas Adiabatic Processes
  \question A frictionless piston-cylinder device contains 0.2 kg of air at 100 kPa and 27°C. The air is now compressed adiabatically until it reaches a final temperature of 77°C.

  \begin{questionparts}
  \item Sketch the P-V diagram of the process with respect to the relevant constant temperature lines, and indicate the work done on this diagram.
    
  \item Using the basic definition of boundary work done determine the boundary work done during the process \answer{[-7.18 kJ]}.
    
  \item Using the energy equation determine the heat transferred during the process \answer{[0 kJ]}, and verify that the process is in fact adiabatic.
  \end{questionparts}
  %
  Use values of specific heat capacity defined at 300K for the entire process.

  % Section 3.7 - Otto Cycle
  \question This is an extension of Example \ref{ex:ch3_otto}, in which we wish to use throughout all four processes the nominal standard specific heat capacity values for air at 300K. Using the values $c_v$ = 0.717 kJ/kgK and $\gamma$ = 1.4, determine:
  \begin{questionparts}
  \item the temperature and pressure of the air at the end of each process \answer{ [$p_2$ = 1838 kPa, $T_2$ = 689K, $T_3$ = 1805K, $p_3$ = 4815 kPa, $p_4$ = 262 kPa, $T_4$ = 786K] }

  \item  the net work output/cycle \answer{ [451.5 kJ/kg]}, and

  \item  the thermal efficiency of this engine cycle. \answer{ [$\eta_{th}$ = 56\%]}
  \end{questionparts}
  \newpage
  %
  % Section 3.8 - Diesel Cycle
  \question Consider the expansion stroke of a typical Air Standard Diesel cycle engine which has a compression ratio of 20 and a cutoff ratio of 2. At the beginning of the process (fuel injection) the initial temperature is 627°C, and the air expands at a constant pressure of 6.2 MPa until cutoff. Subsequently the air expands adiabatically (no heat transfer) until it reaches the maximum volume.
\begin{questionparts}
\item Sketch this process on a $p$-$v$ diagram showing clearly all three states. Shade on the diagram the total work done during the entire expansion process.
\item Determine the temperatures reached at the end of the constant pressure (fuel injection) process \answer{ [1800K]}, as well as at the end of the expansion process \answer{ [830K]}, and draw the three relevant constant temperature lines on the $p$-$v$ diagram.
\item Determine the total work done during the expansion stroke \answer{ [1087 kJ/kg]}.
\item Determine the total heat supplied to the air during the expansion stroke \answer{ [1028 kJ/kg]}.
\item Using Google Colab and CoolProp, plot the three states and two processes on a $p$-$v$ diagram.  Use the methodology in Section \ref{sec:plottingCurvedLines} to plot the curved expansion process.
\end{questionparts}
Use the specific heat values defined at 1000K for the entire expansion process, obtained from the table of Specific Heat Capacities of Air.

\question For an ideal Diesel engine with $r_c = 2$, what does the compression ratio need to be to match the efficiency of an ideal Otto engine with a compression ratio of $r=8$?

\end{homework}
