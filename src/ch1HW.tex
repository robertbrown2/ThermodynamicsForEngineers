
\begin{homework}
%--------------------------------------------------------------------
\question How are states, processes, and cycles related?
%--------------------------------------------------------------------
\question Describe the difference between an intensive and extensive property.
%--------------------------------------------------------------------
\question For each of the following properties, state whether it is intrinsic or extrinsic.
\begin{itemize}
\item Pressure
\item Volume
\item Kinetic energy
\item Specific internal energy
\end{itemize}
%--------------------------------------------------------------------
\question For each of the following systems, state whether it is open, closed, and/or isolated.
\begin{itemize}
\item The ice and water inside of a well-insulated cooler
\item An airtight piston, which is heated from the outside
\item A water spigot
\end{itemize}
%--------------------------------------------------------------------
\question Describe the difference between an isothermal process and an adiabatic process.

%--------------------------------------------------------------------
\question A soda can has a diameter of 2.13 inches and a height of 4.83 inches.  Additionally, it has a mass of 384 g.  Assuming the can is perfectly cylindrical, and the mass of aluminum is 14.7 g, determine:
\begin{itemize}
\item The volume of soda in the can (in ${\rm ft^3}$). \answer{[0.0100 ${\rm ft^3}$]}
\item The mass of soda in the can (in lbm). \answer{[0.812 lbm]}
\item The density of soda in the can (in $\frac{\rm kg}{\rm m^3}$) \answer{[1304 $\frac{\rm kg}{\rm m^3}$]}
\end{itemize}
%--------------------------------------------------------------------
\question The temperature in Chicago in winter can be as low as 14°F. What is the temperature in °C, K, and °R?
%--------------------------------------------------------------------
\question A piston with a diameter of 3'' compresses air, which is at an {\bf absolute} pressure of 2 atm.  What is the net force required to hold the piston still? Provide your answer in lbf. \answer{[103.9 lbf]}


Hint: don't forget atmosphere ($p_{atm}=1 \rm\ atm$) pushing on the other side.
%--------------------------------------------------------------------
\question Rework the previous question in Google Colab.  Plot the force required to hold the piston still, while varying the diameter of the piston between 1'' and 10''.  Plot your result, including a title and labels on both the x- and y- axes.
%--------------------------------------------------------------------
\question A barometer measures 690 mmHg.  What is the atmospheric pressure in kPa? \answer{[92 kPa]}
%--------------------------------------------------------------------
\question A manometer connects a tank with an unknown pressure to atmosphere.  If the manometer is filled with water at 20°C ($\rho = 1000 {\rm kg/m^3}$), and a height difference of 20 cm is measured, what is the gauge pressure in Pa?  \answer{[1962 Pa]}  If the atmospheric pressure is 101 kPa, what is the absolute pressure in Pa? \answer{[102.96 kPa]}
%--------------------------------------------------------------------
\end{homework}
