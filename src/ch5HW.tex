\begin{homework}
  % Section 5.1 Conceptual
  \question In your own words, describe entropy.

  \question A gas has a temperature of 500 K.  Would you expect the entropy to be higher if the gas has a pressure of 100 kPa or 5 MPa?

%  \question Define a microstate.  How does the concept of microstates relate to entropy?

  % Section 5.2 Conceptual
%  \question Give an example of a reversible process.  How about an irreversible process?
  \question What is the difference between reversible and irreversible processes in terms of entropy change?
  \question Can the entropy in a closed system decrease?  What about an isolated system?
  % Section 5.3 Conceptual
  \question Which properties are most natural from a molecular view of matter?
  \question How can we transfer entropy?
%  \question How are pressure and temperature similar?
  % Section 5.4 Conceptual
  \question What is a reservoir in thermodynamics?
%  \question How does the Clausius Statement relate to the simple form of the Second Law?
  \question How do the Clausius and Kelvin-Planck Statements relate to one another?

  % Section 5.5 Conceptual
  \question Which of the following would increase the Carnot efficiency?
  \begin{questionparts}
  \item Increasing the temperature of the hot reservoir.
  \item Decreasing the temperature of the hot reservoir.
  \item Increasing the amount of heat transferred from the hot reservoir.
  \item Decreasing the amount of heat transferred to the cold reservoir.
  \item Decreasing the temperature of the cold reservoir.
  \end{questionparts}
  
  % Section 5.1 Work-out
  \question Using CoolProp or tables, determine the entropy of the following states:
  \begin{questionparts}
  \item Water at a pressure of 100 kPa and a quality of 1.0. \answer{[7.36 $\frac{\rm kJ}{\rm kg\,K}$]}
  \item R134a at a pressure of 1 MPa and a temperature of 130°C. \answer{[1.98 $\frac{\rm kJ}{\rm kg\,K}$]}
  \item Air at a pressure of 10 kPa and a temperature of 200 K. \answer{[4.15 $\frac{\rm kJ}{\rm kg\,K}$]}
  \item Methane (use ``CH4'' or ``methane'' in CoolProp) at a pressure of 100 kPa and an enthalpy of 200 kJ/kg. \answer{[1.79 $\frac{\rm kJ}{\rm kg\,K}$]}
  \end{questionparts}
  % Section 5.2 Work-out
  % Section 5.3 Work-out
  \question Air at 1 MPa and 200 K is compressed in a piston adiabatically and reversibly to a pressure of 2 MPa.  Determine the final temperature using:
  \begin{questionparts}
  \item The ideal gas adiabatic relations, with $\gamma=1.4$. \answer{[243.8 K]}
  \item CoolProp. \answer{[244.8 K]}
  \end{questionparts}

  \question Heat is transferred into 1 kg of liquid water at a constant pressure of 100 kPa and a quality of 0 until it is fully boiled.
  \begin{questionparts}
  \item How much heat transfer is necessary to boil the water? \answer{[2257.4 kJ]}
  \item How much entropy is transferred to the water through heat transfer? \answer{[6.056 kJ/K]}
  \item Based off the equation $\Delta s = \frac{Q}{T}$, what is the temperature that boiling occurred at? \answer{[372.75 K]}
  \end{questionparts}
  
  % Section 5.4 Work-out
  % Section 5.5 Work-out
  \question A heat pump is used to meet the heating requirements of a house and maintain it at 20°C. On a day when the outdoor air temperature drops to -10°C it is estimated that the house looses heat at the rate of 10 kW. Under these conditions the actual Coefficient of Performance ($COP_{HP}$) of the heat pump is 2.5.
  \begin{questionparts}
  \item Draw a diagram representing the heat pump system showing the flow of energy and the temperatures.
  \item Determine the actual power consumed by the heat pump \answer{[4 kW]}.
  \item Determine the power that would be consumed by a reversible heat pump under these conditions \answer{[1.02 kW]}.
  \item Determine the power that would be consumed by an electric resistance heater under these conditions \answer{[10 kW]}
  \item Is the performance of the actual heat pump possible?
  \end{questionparts}
  \question During an experiment conducted at 25°C, a student measured that a Stirling cycle refrigerator that draws 250W of power has removed 1000kJ of heat from the refrigerated space maintained at -30°C. The running time of the refrigerator during the experiment was 20 min.
  \begin{questionparts}
  \item Draw a diagram representing the refrigerator system showing the flow of energy and the temperatures.
  \item Determine the actual and reversible coefficients of performance \answer{[$\rm COP_R$ = 3.33, $\rm COP_{R,rev}$ = 4.42]}.
  \item Are these values possible?
  \end{questionparts}
\end{homework}
