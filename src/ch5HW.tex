\begin{homework}
  \question A heat pump is used to meet the heating requirements of a house and maintain it at 20°C. On a day when the outdoor air temperature drops to -10°C it is estimated that the house looses heat at the rate of 10 kW. Under these conditions the actual Coefficient of Performance ($COP_{HP}$) of the heat pump is 2.5.
  \begin{questionparts}
  \item Draw a diagram representing the heat pump system showing the flow of energy and the temperatures.
  \item Determine the actual power consumed by the heat pump \answer{[4 kW]}.
  \item Determine the power that would be consumed by a reversible heat pump under these conditions \answer{[1.02 kW]}.
  \item Determine the power that would be consumed by an electric resistance heater under these conditions \answer{[10 kW]}
  \item Determine if the performance of the actual heat pump is feasible.
  \end{questionparts}
  \question During an experiment conducted at 25°C, a student measured that a Stirling cycle refrigerator that draws 250W of power has removed 1000kJ of heat from the refrigerated space maintained at -30°C. The running time of the refrigerator during the experiment was 20 min.
  \begin{questionparts}
  \item Draw a diagram representing the refrigerator system showing the flow of energy and the temperatures.
  \item Determine the actual and reversible coefficients of performance \answer{[COPR = 3.33, COPR,rev = 4.42]}.
  \item Determine if these measurements are reasonable. \answer{[$COP_R/COP_{R,rev}$ = 75\% > 60\% - not feasible]}. State the reasons for your conclusions. 
  \end{questionparts}
\end{homework}
