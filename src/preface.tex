This work is an attempt to bring high-quality Open Educational Resources to Engineering Thermodynamics at Abilene Christian University, as supported by the Alternative Textbook Initiative Grant.

The goal of Open Educational Resources is to combat the increased cost of textbooks seen in recent years.  Additionally, having resources that are completely free allows students to maintain a library of books, and encourages students to access the book, when they may not have otherwise purchased it.

Another benefit is the ability of the professor to curate exactly what is provided to the student.  Many textbooks are padded with chapters that are skipped in the course, often leading to confusion with homeworks and textbook readings.

Finally, OER allows for collaboration and continuous improvement.  As you work through this text, be on the lookout for things that are confusing, misleading, or simply poorly explained.  Mistakes and errors can be fixed this semester, and any feedback can go into the next version of the book much faster than a traditional publisher.

This text is an adaptation of a web resource originally developed by Dr. Israel Urieli of Ohio University, titled \href{https://www.ohio.edu/mechanical/thermo/}{Engineering Thermodynamics - A Graphical Approach}.  The resource is unique for a number of reasons, the greatest of which being that it is licensed under a \href{https://creativecommons.org/licenses/by-nc-sa/3.0/us/}{Creative Commons license} which allows free sharing and adaptation by other parties.  I should also mention Dr. Diana Bairaktarova, whose adaptation of Dr. Urieli's work was also invaluable in the process of building this version.
